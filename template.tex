% Options for packages loaded elsewhere
\PassOptionsToPackage{unicode}{hyperref}
\PassOptionsToPackage{hyphens}{url}
\PassOptionsToPackage{dvipsnames,svgnames,x11names}{xcolor}
%
\documentclass[
  letterpaper,
  DIV=11,
  numbers=noendperiod,
  oneside]{scrartcl}

\usepackage{amsmath,amssymb}
\usepackage{iftex}
\ifPDFTeX
  \usepackage[T1]{fontenc}
  \usepackage[utf8]{inputenc}
  \usepackage{textcomp} % provide euro and other symbols
\else % if luatex or xetex
  \usepackage{unicode-math}
  \defaultfontfeatures{Scale=MatchLowercase}
  \defaultfontfeatures[\rmfamily]{Ligatures=TeX,Scale=1}
\fi
\usepackage{lmodern}
\ifPDFTeX\else  
    % xetex/luatex font selection
\fi
% Use upquote if available, for straight quotes in verbatim environments
\IfFileExists{upquote.sty}{\usepackage{upquote}}{}
\IfFileExists{microtype.sty}{% use microtype if available
  \usepackage[]{microtype}
  \UseMicrotypeSet[protrusion]{basicmath} % disable protrusion for tt fonts
}{}
\makeatletter
\@ifundefined{KOMAClassName}{% if non-KOMA class
  \IfFileExists{parskip.sty}{%
    \usepackage{parskip}
  }{% else
    \setlength{\parindent}{0pt}
    \setlength{\parskip}{6pt plus 2pt minus 1pt}}
}{% if KOMA class
  \KOMAoptions{parskip=half}}
\makeatother
\usepackage{xcolor}
\usepackage[left=1in,marginparwidth=2.0666666666667in,textwidth=4.1333333333333in,marginparsep=0.3in]{geometry}
\setlength{\emergencystretch}{3em} % prevent overfull lines
\setcounter{secnumdepth}{-\maxdimen} % remove section numbering
% Make \paragraph and \subparagraph free-standing
\ifx\paragraph\undefined\else
  \let\oldparagraph\paragraph
  \renewcommand{\paragraph}[1]{\oldparagraph{#1}\mbox{}}
\fi
\ifx\subparagraph\undefined\else
  \let\oldsubparagraph\subparagraph
  \renewcommand{\subparagraph}[1]{\oldsubparagraph{#1}\mbox{}}
\fi


\providecommand{\tightlist}{%
  \setlength{\itemsep}{0pt}\setlength{\parskip}{0pt}}\usepackage{longtable,booktabs,array}
\usepackage{calc} % for calculating minipage widths
% Correct order of tables after \paragraph or \subparagraph
\usepackage{etoolbox}
\makeatletter
\patchcmd\longtable{\par}{\if@noskipsec\mbox{}\fi\par}{}{}
\makeatother
% Allow footnotes in longtable head/foot
\IfFileExists{footnotehyper.sty}{\usepackage{footnotehyper}}{\usepackage{footnote}}
\makesavenoteenv{longtable}
\usepackage{graphicx}
\makeatletter
\def\maxwidth{\ifdim\Gin@nat@width>\linewidth\linewidth\else\Gin@nat@width\fi}
\def\maxheight{\ifdim\Gin@nat@height>\textheight\textheight\else\Gin@nat@height\fi}
\makeatother
% Scale images if necessary, so that they will not overflow the page
% margins by default, and it is still possible to overwrite the defaults
% using explicit options in \includegraphics[width, height, ...]{}
\setkeys{Gin}{width=\maxwidth,height=\maxheight,keepaspectratio}
% Set default figure placement to htbp
\makeatletter
\def\fps@figure{htbp}
\makeatother

\usepackage{booktabs}
\usepackage{longtable}
\usepackage{array}
\usepackage{multirow}
\usepackage{wrapfig}
\usepackage{float}
\usepackage{colortbl}
\usepackage{pdflscape}
\usepackage{tabu}
\usepackage{threeparttable}
\usepackage{threeparttablex}
\usepackage[normalem]{ulem}
\usepackage{makecell}
\usepackage{xcolor}
\usepackage{caption}
\KOMAoption{captions}{tableheading}
\makeatletter
\makeatother
\makeatletter
\makeatother
\makeatletter
\@ifpackageloaded{caption}{}{\usepackage{caption}}
\AtBeginDocument{%
\ifdefined\contentsname
  \renewcommand*\contentsname{Table of contents}
\else
  \newcommand\contentsname{Table of contents}
\fi
\ifdefined\listfigurename
  \renewcommand*\listfigurename{List of Figures}
\else
  \newcommand\listfigurename{List of Figures}
\fi
\ifdefined\listtablename
  \renewcommand*\listtablename{List of Tables}
\else
  \newcommand\listtablename{List of Tables}
\fi
\ifdefined\figurename
  \renewcommand*\figurename{Figure}
\else
  \newcommand\figurename{Figure}
\fi
\ifdefined\tablename
  \renewcommand*\tablename{Table}
\else
  \newcommand\tablename{Table}
\fi
}
\@ifpackageloaded{float}{}{\usepackage{float}}
\floatstyle{ruled}
\@ifundefined{c@chapter}{\newfloat{codelisting}{h}{lop}}{\newfloat{codelisting}{h}{lop}[chapter]}
\floatname{codelisting}{Listing}
\newcommand*\listoflistings{\listof{codelisting}{List of Listings}}
\makeatother
\makeatletter
\@ifpackageloaded{caption}{}{\usepackage{caption}}
\@ifpackageloaded{subcaption}{}{\usepackage{subcaption}}
\makeatother
\makeatletter
\@ifpackageloaded{tcolorbox}{}{\usepackage[skins,breakable]{tcolorbox}}
\makeatother
\makeatletter
\@ifundefined{shadecolor}{\definecolor{shadecolor}{rgb}{.97, .97, .97}}
\makeatother
\makeatletter
\makeatother
\makeatletter
\ifdefined\Shaded\renewenvironment{Shaded}{\begin{tcolorbox}[borderline west={3pt}{0pt}{shadecolor}, interior hidden, enhanced, frame hidden, boxrule=0pt, sharp corners, breakable]}{\end{tcolorbox}}\fi
\makeatother
\makeatletter
\@ifpackageloaded{sidenotes}{}{\usepackage{sidenotes}}
\@ifpackageloaded{marginnote}{}{\usepackage{marginnote}}
\makeatother
\makeatletter
\makeatother
\ifLuaTeX
  \usepackage{selnolig}  % disable illegal ligatures
\fi
\IfFileExists{bookmark.sty}{\usepackage{bookmark}}{\usepackage{hyperref}}
\IfFileExists{xurl.sty}{\usepackage{xurl}}{} % add URL line breaks if available
\urlstyle{same} % disable monospaced font for URLs
\hypersetup{
  pdftitle={Recommendations},
  pdfauthor={Joey Trampush},
  colorlinks=true,
  linkcolor={blue},
  filecolor={Maroon},
  citecolor={Blue},
  urlcolor={Blue},
  pdfcreator={LaTeX via pandoc}}

\title{NEUROCOGNITIVE EXAMINATION}
\author{}
\date{}

\begin{document}
\maketitle
\hypertarget{tests-administered}{%
\section{TESTS ADMINISTERED}\label{tests-administered}}

\hypertarget{neurobehavioral-status-exam}{%
\section{NEUROBEHAVIORAL STATUS
EXAM}\label{neurobehavioral-status-exam}}

\hypertarget{referral}{%
\subsection{Referral}\label{referral}}

Ms.~XXXX was referred for neuropsychological testing as part of a
comprehensive presurgical work-up of her epilepsy~syndrome. The results
will be used in surgical and treatment planning

\hypertarget{background}{%
\subsection{Background}\label{background}}

The following information was obtained during an interview with Ms.~XXXX
and from review of available medical records. Ms.~XXXX has a history of
seizures since age 14. While they were initially well controlled, in the
last 3 years they have become refractory to multiple seizure
medications. She currently experiences one to two seizures per week.
Seizures are characterized by behavioral and speech arrest, and last
approximately 2 minutes. Postictally, she is mildly confused with
language disruption, but she returns to baseline in approximately 10
minutes. She is treated with Depakote, Keppra, and Lamictal.

Cognitive complaints: mild memory and word-finding issues over the last
2 years.

\hypertarget{past-neuropsychological-test-results}{%
\subsubsection{Past Neuropsychological Test
Results}\label{past-neuropsychological-test-results}}

Patient denied having prior testing.

\hypertarget{other-medical-history}{%
\subsubsection{Other Medical History}\label{other-medical-history}}

\begin{itemize}
\tightlist
\item
  Frequent sinus infections.
\item
  Other medications: Femcon, folic acid.
\item
  Appetite/weight: Normal, no changes.
\item
  Sleep: Normal, no changes.
\item
  Alcohol/tobacco: No history of abuse; denied current use.
\end{itemize}

\hypertarget{surgical-history}{%
\subsubsection{Surgical History}\label{surgical-history}}

Sinus surgery.

\hypertarget{psychiatric-history}{%
\subsubsection{Psychiatric History}\label{psychiatric-history}}

Patient denied.

\hypertarget{family-history}{%
\subsubsection{Family History}\label{family-history}}

Patient denied a family history of neurological conditions. Her brother
was diagnosed with ADHD.

\hypertarget{culturalsocial-background}{%
\subsubsection{Cultural/Social
Background}\label{culturalsocial-background}}

Ms.~XXXX is a Caucasian female who was born and raised in XXXX. She is
single, never married, and has no children.

\hypertarget{educational-history}{%
\subsubsection{Educational History}\label{educational-history}}

Ms.~XXXX graduated from college and completed some Master's-level
courses. She did not endorse any difficulties with advancing through
school.

\hypertarget{occupational-history}{%
\subsubsection{Occupational History}\label{occupational-history}}

Ms.~XXXX has worked full time as an administrator in a small company.
She has maintained this job for the past 6 years.

\hypertarget{neurocognitive-findings}{%
\section{NEUROCOGNITIVE FINDINGS}\label{neurocognitive-findings}}

\hypertarget{behavioral-observations}{%
\subsection{Behavioral Observations}\label{behavioral-observations}}

\begin{itemize}
\tightlist
\item
  \emph{Appearance:} Appropriate grooming and dress for context.
\item
  \emph{Behavior/attitude:} Cooperative, engaged.
\item
  \emph{Speech/language:} Fluent and normal in rate, volume, and
  prosody.
\item
  \emph{Mood/affect:} Neutral, range was full and appropriate.
\item
  \emph{Sensory/motor:} Performance was not limited by any obvious
  sensory or motor difficulties.
\item
  \emph{Cognitive process:} Coherent and goal directed.
\item
  \emph{Motivation/effort:} Normal.
\end{itemize}

\hypertarget{intelligencegeneral-ability}{%
\subsection{Intelligence/General
Ability}\label{intelligencegeneral-ability}}

\setlength{\LTpost}{0mm}
\begin{longtable*}{l|ccc}
\toprule
\multicolumn{1}{l}{} & \textbf{Score} & \textbf{‰ Rank} & \textbf{Range} \\ 
\midrule
\multicolumn{4}{l}{Composite Scores} \\ 
\midrule
\hspace*{10px} Cognitive Proficiency & 132 & 98.0 & Exceptionally High \\ 
\hspace*{10px} Crystallized Knowledge & 115 & 84.0 & High Average \\ 
\hspace*{10px} Fluid Reasoning & 118 & 88.0 & High Average \\ 
\hspace*{10px} Processing Speed & 96 & 39.0 & Average \\ 
\hspace*{10px} Working Memory & 132 & 98.0 & Exceptionally High \\ 
\midrule
\multicolumn{4}{l}{NAB} \\ 
\midrule
\hspace*{10px} NAB Attention Index & 112 & 79.0 & High Average \\ 
\hspace*{10px} NAB Executive Functions Index & 106 & 66.0 & Average \\ 
\hspace*{10px} NAB Language Index & 99 & 47.0 & Average \\ 
\hspace*{10px} NAB Memory Index & 96 & 39.0 & Average \\ 
\hspace*{10px} NAB Spatial Index & 121 & 92.0 & Above Average \\ 
\hspace*{10px} NAB Total Index & 110 & 75.0 & High Average \\ 
\midrule
\multicolumn{4}{l}{WRAT-5} \\ 
\hspace*{10px} Word Reading & 113 & 80.7 & High Average \\ 
\bottomrule
\end{longtable*}
\begin{minipage}{\linewidth}
\emph{Note:} Index scores have a mean of 100 and a standard deviation of 15.\\
\end{minipage}

\begin{figure}

{\centering \includegraphics{template_files/figure-pdf/fig-iq-1.pdf}

}

\caption{\label{fig-iq}\emph{General Ability} refers to an overall
capacity to reason, to solve problems, and to learn useful information.
\emph{Crystallized Knowledge} refers to the ability to learn and use
language to reason and understand how the world works. \emph{Fluid
Reasoning} refers to the ability to use logical reasoning to figure
things out without being told exactly how things work, analyze and solve
novel problems, identify patterns and relationships that underpin these
problems, and apply logic.}

\end{figure}

\hypertarget{academic-skills}{%
\subsection{Academic Skills}\label{academic-skills}}

\hypertarget{verballanguage}{%
\subsection{Verbal/Language}\label{verballanguage}}

\setlength{\LTpost}{0mm}
\begin{longtable*}{l|ccc}
\toprule
\multicolumn{1}{l}{} & \textbf{Score} & \textbf{‰ Rank} & \textbf{Range} \\ 
\midrule
\multicolumn{4}{l}{D-KEFS} \\ 
\midrule
\hspace*{10px} D-KEFS Color Naming & 8 & 25.2 & Average \\ 
\hspace*{10px} D-KEFS Word Reading & 11 & 63.1 & Average \\ 
\midrule
\multicolumn{4}{l}{NAB} \\ 
\midrule
\hspace*{10px} Auditory Comprehension & 51 & 54.0 & Average \\ 
\hspace*{10px} Naming & 55 & 69.0 & Average \\ 
\midrule
\multicolumn{4}{l}{NIH EXAMINER} \\ 
\midrule
\hspace*{10px} Category Fluency & 58 & 78.8 & High Average \\ 
\hspace*{10px} Letter Fluency & 68 & 96.4 & Above Average \\ 
\bottomrule
\end{longtable*}
\begin{minipage}{\linewidth}
\emph{Note:} T-scores have a mean of 50 and a standard deviation of 10. Scaled scores have a mean of 10 and a standard deviation of 3.\\
\end{minipage}

\begin{figure}

{\centering \includegraphics{template_files/figure-pdf/fig-verbal-1.pdf}

}

\caption{\label{fig-verbal}Verbal/Language refers to the ability to
access and apply acquired word knowledge, to verbalize meaningful
concepts, to understand complex multistep instructions, to think about
verbal information, and to express oneself using words.}

\end{figure}

\hypertarget{visual-perceptionconstruction}{%
\subsection{Visual
Perception/Construction}\label{visual-perceptionconstruction}}

\setlength{\LTpost}{0mm}
\begin{longtable*}{l|ccc}
\toprule
\multicolumn{1}{l}{} & \textbf{Score} & \textbf{‰ Rank} & \textbf{Range} \\ 
\midrule
\multicolumn{4}{l}{NAB} \\ 
\midrule
\hspace*{10px} Design Construction & 62 & 88.0 & High Average \\ 
\hspace*{10px} Visual Discrimination & 59 & 82.0 & High Average \\ 
\midrule
\multicolumn{4}{l}{Rey Complex Figure} \\ 
\hspace*{10px} ROCFT Copy & 54 & 65.5 & Average \\ 
\bottomrule
\end{longtable*}
\begin{minipage}{\linewidth}
\emph{Note:} T-scores have a mean of 50 and a standard
deviation of 10.
Scaled scores have a mean of 10 and a standard deviation of
3.\\
\end{minipage}

\hypertarget{attentionexecutive}{%
\subsection{Attention/Executive}\label{attentionexecutive}}

\setlength{\LTpost}{0mm}
\begin{longtable*}{l|ccc}
\toprule
\multicolumn{1}{l}{} & \textbf{Score} & \textbf{‰ Rank} & \textbf{Range} \\ 
\midrule
\multicolumn{4}{l}{NAB} \\ 
\midrule
\hspace*{10px} Digits Backward & 67 & 96.0 & Above Average \\ 
\hspace*{10px} Digits Backward Longest Span & – & 90.0 & High Average \\ 
\hspace*{10px} Digits Forward & 70 & 98.0 & Exceptionally High \\ 
\hspace*{10px} Digits Forward Longest Span & – & 90.0 & High Average \\ 
\hspace*{10px} Mazes & 43 & 24.0 & Low Average \\ 
\hspace*{10px} Numbers \& Letters Part A Efficiency & 44 & 27.0 & Average \\ 
\hspace*{10px} Numbers \& Letters Part B Efficiency & 42 & 21.0 & Low Average \\ 
\hspace*{10px} Word Generation & 65 & 93.0 & Above Average \\ 
\hspace*{10px} Word Generation Perseverations & – & 50.0 & Average \\ 
\midrule
\multicolumn{4}{l}{Trail Making Test} \\ 
\midrule
\hspace*{10px} TMT, Part A & 44 & 27.4 & Average \\ 
\hspace*{10px} TMT, Part B & 51 & 54.0 & Average \\ 
\midrule
\multicolumn{4}{l}{WAIS-IV} \\ 
\hspace*{10px} Coding & 13 & 84.0 & High Average \\ 
\bottomrule
\end{longtable*}
\begin{minipage}{\linewidth}
\emph{Note:} T-scores have a mean of 50 and a standard deviation of 10. Scaled scores have a mean of 10 and a standard deviation of 3.\\
\end{minipage}

\begin{figure}

{\centering \includegraphics[width=0.7\textwidth,height=\textheight]{template_files/figure-pdf/fig-executive-1.pdf}

}

\caption{\label{fig-executive}Attentional and executive functions
underlie most, if not all, domains of cognitive performance. These are
behaviors and skills that allow individuals to successfully carry-out
instrumental and social activities, academic work, engage with others
effectively, problem solve, and successfully interact with the
environment to get needs met.}

\end{figure}

\hypertarget{memory}{%
\subsection{Memory}\label{memory}}

\setlength{\LTpost}{0mm}
\begin{longtable*}{l|ccc}
\toprule
\multicolumn{1}{l}{} & \textbf{Score} & \textbf{‰ Rank} & \textbf{Range} \\ 
\midrule
\multicolumn{4}{l}{NAB} \\ 
\midrule
\hspace*{10px} List Learning Immediate Recall & 50 & 50.0 & Average \\ 
\hspace*{10px} List Learning Long Delayed Recall & 57 & 76.0 & High Average \\ 
\hspace*{10px} List Learning Short Delayed Recall & 63 & 90.0 & High Average \\ 
\hspace*{10px} Shape Learning Delayed Recognition & 46 & 34.0 & Average \\ 
\hspace*{10px} Shape Learning Immediate Recognition & 45 & 31.0 & Average \\ 
\hspace*{10px} Shape Learning Percent Retention & – & 50.0 & Average \\ 
\hspace*{10px} Story Learning Delayed Recall & 54 & 66.0 & Average \\ 
\hspace*{10px} Story Learning Immediate Recall & 51 & 54.0 & Average \\ 
\hspace*{10px} Story Learning Percent Retention & – & 75.0 & High Average \\ 
\midrule
\multicolumn{4}{l}{Rey Complex Figure} \\ 
\hspace*{10px} ROCFT Delayed Recall & 52 & 57.9 & Average \\ 
\bottomrule
\end{longtable*}
\begin{minipage}{\linewidth}
\emph{Note:} T-scores have a mean of 50 and a standard deviation of 10. Scaled scores have a mean of 10 and a standard deviation of 3.\\
\end{minipage}

\begin{figure*}

{\centering \includegraphics{template_files/figure-pdf/fig-memory-1.pdf}

}

\caption{\label{fig-memory}\emph{Learning and memory} refer to the rate
and ease with which new information (e. g., facts, stories, lists,
faces, names) can be encoded, stored, and later recalled from long-term
memory.}

\end{figure*}

\hypertarget{adhdexecutive-functioning}{%
\subsection{ADHD/Executive
Functioning}\label{adhdexecutive-functioning}}

\setlength{\LTpost}{0mm}
\begin{longtable*}{l|ccc}
\toprule
\multicolumn{1}{l}{} & \textbf{Score} & \textbf{‰ Rank} & \textbf{Range} \\ 
\midrule
\multicolumn{4}{l}{CAARS Observer-Report} \\ 
\midrule
\hspace*{10px} ADHD Index & 35 & 6 & Below Average \\ 
\hspace*{10px} DSM-5 ADHD Symptoms Total & 35 & 6 & Below Average \\ 
\hspace*{10px} DSM-5 Hyperactive-Impulsive Symptoms & 37 & 9 & Low Average \\ 
\hspace*{10px} DSM-5 Inattentive Symptoms & 35 & 6 & Below Average \\ 
\hspace*{10px} Hyperactivity/Restlessness & 42 & 21 & Low Average \\ 
\hspace*{10px} Impulsivity/Emotional Lability & 38 & 11 & Low Average \\ 
\hspace*{10px} Inattention/Memory Problems & 36 & 8 & Below Average \\ 
\hspace*{10px} Problems with Self-Concept & 38 & 11 & Low Average \\ 
\midrule
\multicolumn{4}{l}{CAARS Self-Report} \\ 
\midrule
ADHD Index & 66 & 94 & Above Average \\ 
DSM-5 ADHD Symptoms Total & 63 & 90 & High Average \\ 
DSM-5 Hyperactive-Impulsive Symptoms & 42 & 21 & Low Average \\ 
DSM-5 Inattentive Symptoms & 80 & 99 & Exceptionally High \\ 
Hyperactivity/Restlessness & 43 & 24 & Low Average \\ 
Impulsivity/Emotional Lability & 46 & 34 & Average \\ 
Inattention/Memory Problems & 71 & 98 & Exceptionally High \\ 
Problems with Self-Concept & 71 & 98 & Exceptionally High \\ 
\midrule
\multicolumn{4}{l}{CEFI Observer-Report} \\ 
\midrule
\hspace*{10px} Attention & 131 & 98 & Exceptionally High \\ 
\hspace*{10px} Emotion Regulation & 133 & 99 & Exceptionally High \\ 
\hspace*{10px} Flexibility & 124 & 95 & Above Average \\ 
\hspace*{10px} Full Scale & 130 & 98 & Exceptionally High \\ 
\hspace*{10px} Inhibitory Control & 130 & 98 & Exceptionally High \\ 
\hspace*{10px} Initiation & 125 & 95 & Above Average \\ 
\hspace*{10px} Organization & 126 & 96 & Above Average \\ 
\hspace*{10px} Planning & 121 & 92 & Above Average \\ 
\hspace*{10px} Self-Monitoring & 130 & 98 & Exceptionally High \\ 
\hspace*{10px} Working Memory & 129 & 97 & Above Average \\ 
\midrule
\multicolumn{4}{l}{CEFI Self-Report} \\ 
\midrule
Attention & 79 & 8 & Below Average \\ 
Emotion Regulation & 84 & 14 & Low Average \\ 
Flexibility & 79 & 8 & Below Average \\ 
Full Scale & 73 & 4 & Below Average \\ 
Inhibitory Control & 78 & 7 & Below Average \\ 
Initiation & 67 & 1 & Exceptionally Low \\ 
Organization & 79 & 8 & Below Average \\ 
Planning & 61 & 1 & Exceptionally Low \\ 
Self-Monitoring & 76 & 5 & Below Average \\ 
Working Memory & 85 & 16 & Low Average \\ 
\bottomrule
\end{longtable*}
\begin{minipage}{\linewidth}
\emph{Note:} CAARS Standard scores have a mean of 50 and a standard deviation of 10, and higher scores reflect reduced functioning. CEFI Standard scores have a mean of 100 and a standard deviation of 15, and lower scores reflect reduced functioning.\\
\end{minipage}

\begin{figure}

{\centering \includegraphics{template_files/figure-pdf/fig-adhd-1.pdf}

}

\caption{\label{fig-adhd}Attention and executive functions are
multidimensional concepts that contain several related processes. Both
concepts require self-regulatory skills and have some common
subprocesses; therefore, it is common to treat them together, or even to
refer to both processes when talking about one or the other.}

\end{figure}

\hypertarget{emotionalbehavioralpersonality}{%
\subsection{Emotional/Behavioral/Personality}\label{emotionalbehavioralpersonality}}

\begin{figure}

{\centering \includegraphics{template_files/figure-pdf/fig-emotion-1.pdf}

}

\caption{\label{fig-emotion}Mood/Self-Report}

\end{figure}

\begin{longtable}[]{@{}
  >{\raggedright\arraybackslash}p{(\columnwidth - 6\tabcolsep) * \real{0.3768}}
  >{\centering\arraybackslash}p{(\columnwidth - 6\tabcolsep) * \real{0.1594}}
  >{\centering\arraybackslash}p{(\columnwidth - 6\tabcolsep) * \real{0.1739}}
  >{\centering\arraybackslash}p{(\columnwidth - 6\tabcolsep) * \real{0.2899}}@{}}
\caption{Mood/Self-Report}\tabularnewline
\toprule\noalign{}
\begin{minipage}[b]{\linewidth}\raggedright
\textbf{Scale}
\end{minipage} & \begin{minipage}[b]{\linewidth}\centering
\textbf{Score}
\end{minipage} & \begin{minipage}[b]{\linewidth}\centering
\textbf{‰ Rank}
\end{minipage} & \begin{minipage}[b]{\linewidth}\centering
\textbf{Range}
\end{minipage} \\
\midrule\noalign{}
\endfirsthead
\toprule\noalign{}
\begin{minipage}[b]{\linewidth}\raggedright
\textbf{Scale}
\end{minipage} & \begin{minipage}[b]{\linewidth}\centering
\textbf{Score}
\end{minipage} & \begin{minipage}[b]{\linewidth}\centering
\textbf{‰ Rank}
\end{minipage} & \begin{minipage}[b]{\linewidth}\centering
\textbf{Range}
\end{minipage} \\
\midrule\noalign{}
\endhead
\bottomrule\noalign{}
\endlastfoot
Somatic Complaints & 67 & 95 & Above Average \\
Conversion & 60 & 84 & High Average \\
Somatization & 59 & 81 & High Average \\
Health Concerns & 73 & 98 & Exceptionally High \\
Anxiety & 69 & 97 & Above Average \\
Cognitive (A) & 71 & 98 & Exceptionally High \\
Affective (A) & 73 & 98 & Exceptionally High \\
Physiological (A) & 58 & 78 & High Average \\
Anxiety-Related Disorders & 75 & 99 & Exceptionally High \\
Obsessive-Compulsive & 62 & 88 & High Average \\
Phobias & 51 & 53 & Average \\
Traumatic Stress & 89 & 99 & Exceptionally High \\
Depression & 72 & 98 & Exceptionally High \\
Cognitive (D) & 64 & 91 & Above Average \\
Affective (D) & 74 & 99 & Exceptionally High \\
Physiological (D) & 67 & 95 & Above Average \\
Mania & 68 & 96 & Above Average \\
Activity Level & 63 & 90 & High Average \\
Grandiosity & 67 & 95 & Above Average \\
Irritability & 62 & 88 & High Average \\
Paranoia & 67 & 95 & Above Average \\
Hypervigilance & 51 & 53 & Average \\
Persecution & 75 & 99 & Exceptionally High \\
Resentment & 66 & 94 & Above Average \\
Schizophrenia & 73 & 98 & Exceptionally High \\
Psychotic Experiences & 50 & 50 & Average \\
Social Detachment & 76 & 99 & Exceptionally High \\
Thought Disorder & 73 & 98 & Exceptionally High \\
Borderline Features & 69 & 97 & Above Average \\
Affective Instability & 63 & 90 & High Average \\
Identity Problems & 71 & 98 & Exceptionally High \\
Negative Relationships & 65 & 93 & Above Average \\
Self-Harm & 60 & 84 & High Average \\
Antisocial Features & 53 & 61 & Average \\
Antisocial Behaviors & 45 & 30 & Average \\
Egocentricity & 72 & 98 & Exceptionally High \\
Stimulus-Seeking & 45 & 30 & Average \\
Aggression & 32 & 3 & Below Average \\
Aggressive Attitude & 34 & 5 & Below Average \\
Verbal Aggression & 31 & 2 & Below Average \\
Physical Aggression & 42 & 21 & Low Average \\
Alcohol Problems & 43 & 24 & Low Average \\
Drug Problems & 42 & 21 & Low Average \\
ALC Estimated Score & 57 & 75 & High Average \\
DRG Estimated Score & 56 & 72 & Average \\
Suicidal Ideation & 43 & 24 & Low Average \\
Stress & 64 & 91 & Above Average \\
Nonsupport & 58 & 78 & High Average \\
Treatment Rejection & 31 & 2 & Below Average \\
Dominance & 44 & 27 & Average \\
Warmth & 42 & 21 & Low Average \\
\end{longtable}

\newpage{}

\hypertarget{summaryimpression}{%
\section{SUMMARY/IMPRESSION}\label{summaryimpression}}

\begin{figure*}

\sidecaption{\label{fig-domain-plot}\emph{Note:} \emph{z}-scores have a
mean of 0 and a standard deviation of 1.}

{\centering \includegraphics{template_files/figure-pdf/fig-domain-plot-1.pdf}

}

\end{figure*}

\hypertarget{summary-and-formulation}{%
\subsection{Summary and Formulation}\label{summary-and-formulation}}

XXXX is a 28-year-old left-handed female with a history of medically
refractory epilepsy who was referred for neuropsychological testing as
part of a comprehensive presurgical work-up. General cognitive ability
is well within normal limits, and there is no evidence of decline from
premorbid estimates. No deficits were detected in the domains of
attention, processing speed, motor functioning, or visuospatial skills.
Although many aspects of executive functioning, language functioning,
and memory were within normal limits, she demonstrated mildly
inefficient problem solving and hypothesis testing, weaknesses in word
retrieval, and inefficiency in new learning of verbal information. There
is no evidence of a mood disorder at this time.

In conclusion, the cognitive profile is mildly localizing and
lateralizing as there are elements that suggest relative left temporal
involvement. That said, there were features seen during testing that
suggested mild frontal systems disruption as well. Although Ms.~XXXX
lives alone, she has arranged to have her mother stay with her for a
week or two after surgery to help her and provide support in the
immediate postacute period. She seems to have a good understanding of
and appropriate expectations for the surgery. Taken together with her
current asymptomatic mood profile, there are no obvious psychological
risk factors for a poor outcome or need for presurgical mental health
intervention.

\hypertarget{diagnostic-impression}{%
\subsection{Diagnostic Impression}\label{diagnostic-impression}}

\begin{itemize}
\tightlist
\item
  315.00 (F81.0) Specific Learning Disorder, With Impairment in Reading
  (word reading accuracy, reading rate or fluency, reading
  comprehension)
\item
  315.2 (F81.81) Specific Learning Disorder, With Impairment in Written
  Expression (spelling accuracy)
\item
  314.01 (F90.9) Unspecified Attention-Deficit/Hyperactivity Disorder
  (ADHD)
\end{itemize}

\hypertarget{recommendations}{%
\section{RECOMMENDATIONS}\label{recommendations}}

\hypertarget{recommendations-for-medicalhealthcare}{%
\subsection{Recommendations for
Medical/Healthcare}\label{recommendations-for-medicalhealthcare}}

\begin{itemize}
\tightlist
\item
  Rec 1
\item
  Rec 2
\item
  Rec 3
\end{itemize}

\hypertarget{recommendations-for-school}{%
\subsection{Recommendations for
School}\label{recommendations-for-school}}

\begin{itemize}
\tightlist
\item
  Rec 1
\item
  Rec 2
\item
  Rec 3
\end{itemize}

\hypertarget{recommendations-for-homefamily}{%
\subsection{Recommendations for
Home/Family}\label{recommendations-for-homefamily}}

\begin{itemize}
\tightlist
\item
  Rec 1
\item
  Rec 2
\item
  Rec 3
\end{itemize}

\hypertarget{recommendations-for-follow-up-evaluation}{%
\subsection{Recommendations for Follow-Up
Evaluation}\label{recommendations-for-follow-up-evaluation}}

\begin{itemize}
\item
  Follow-up assessment is not recommended at this time unless further
  concerns arise that need to be addressed.
\item
  Follow-up assessment in 12-18 months is recommended to gauge
  (ref:first-name)'s progress and to assess the impact of the above
  interventions, unless further concerns arise that need to be addressed
  sooner.
\end{itemize}

It was a pleasure to work with Mr.~Smalls. Please contact me with any
questions or concerns regarding this patient.

Sincerely,

\textbf{Joey W. Trampush, Ph.D.}\\
Assistant Professor of Psychiatry\\
Department of Psychiatry and the Behavioral Sciences\\
University of Southern California Keck School of Medicine\\
CA License PSY29212

\newpage{}

\hypertarget{appendix}{%
\section{APPENDIX}\label{appendix}}

\hypertarget{test-selection-procedures}{%
\subsection{Test Selection Procedures}\label{test-selection-procedures}}

Neuropsychological tests are intrinsically performance-based, and
cognitive performance assessed during this neuropsychological evaluation
is summarized above. Where appropriate, qualitative observations are
included. Cultural considerations were made when selecting measures,
interpreting results, and making diagnostic impressions and
recommendations. Results from formal tests are reported in comparison to
other individuals the same age, sex, and educational level as range of
functioning (e.g., below average, average, above average). Test score
labels are intended solely to be descriptive, identifying positions of
scores relative to a normal curve distribution, and should be
interpreted within the context of the patient's individual presentation
and history. Although standardized scores provide the clinician with an
important and necessary understanding of the patient's test performance
compared with a normative group, they do not on their own lead to
accurate diagnosis or treatment recommendations.

\hypertarget{conversion-of-test-scores}{%
\subsection{Conversion of Test Scores}\label{conversion-of-test-scores}}

\includegraphics[width=4.66in,height=\textheight]{tbl_range.png}



\end{document}
